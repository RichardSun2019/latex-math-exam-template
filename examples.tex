\documentclass[12pt, a4paper]{article}
\usepackage{amsmath, enumitem, fancyhdr, geometry, fontspec, lastpage, titlesec}
\usepackage{xeCJK}
\usepackage{CJKnumb}
\usepackage{bm}
\usepackage{tikz-cd}
\usepackage{tikz}
\usetikzlibrary{calc}
% 页边距
\geometry{
    left=2cm,
    right=2cm,
    top=2cm,
    bottom=2cm
}
% 字体
\setmainfont{Times New Roman}
\setCJKmainfont{"FZShuSong_GB18030-Z01"}
% section格式
\renewcommand{\thesection}{\CJKnumber{\value{section}}、}

% 自定义命令
\def\mycenter#1{\begin{center} \bf \sffamily \large #1 \end{center}}

% 页码
\pagestyle{fancy}
\fancyhf{}
\cfoot{数学试卷\ 第\thepage 页\ 共\pageref{LastPage}\ 页}
\renewcommand{\headrulewidth}{0pt}

% 题目格式. III太宽了,稍作调整
\newlist{question}{enumerate}{2}

% 北京、天津等地,(1), (2), ...
% \setlist[question,1]{
%     label=\protect{\makebox[1.6em]{(\hfill\arabic*\hfill)}},
%     labelwidth=1.6em,
%     labelsep=0.4em,
%     leftmargin=2em,
% }

% 常见格式,1., 2., ...
\setlist[question,1]{
    label=\protect{\makebox[1.6em]{\hfill\arabic*.\hfill}},
    labelwidth=1.6em,
    labelsep=0.4em,
    leftmargin=2em,
}

\setlist[question, 2]{
    topsep=0pt,
    label=\protect{\makebox[1.6em]{(\hfill\Roman*\hfill)}},
    labelwidth=1.6em,
    labelsep=0.4em,
    leftmargin=2em,
}


\begin{document}



\begin{center}
    % \huge 某某学校xxxx-xxxx学年xx考试 \\
    \huge 暑假班中期考试 \\
    \Large \bf \sffamily 数学试卷-平面向量部分
\end{center}

本试卷共\pageref{LastPage}\ 页,150分。考试时长120分钟。测试分三个部分:填空、计算和证明。计算题仅给出最后结果不给分。
% 考生务必将答案答在答题纸上,在试卷上作答无效。考试结束后,将本试卷和答题纸一并交回。

% \mycenter{第一部分(计算题\ 共\textmd{\rmfamily 80}分)}

% \section{计算题:本题共\textmd{\rmfamily 10}小题,每小题\textmd{\rmfamily 4}分,共\textmd{\rmfamily 40}分。在每小题给出的四个选项中,只有一项是符合题目要求的。}
\section{填空题:本题共。。。}
\begin{question}[resume]
\item 设 $P$ 是线段 $P_{1}P_{2}$ 上的一点,点 $P_{1}$、$P_{2}$ 的坐标分别是 $(x_{1},y_{1})$、$(x_{2},y_{2})$。
\begin{question}
    \item 当 $P$ 是线段 $P_{1}P_{2}$ 的中点时,$P$ 的坐标是\underline{\hspace{6em}};
    \item 当$P$ 是线段 $P_{1}P_{2}$ 的一个三等分点时,点 $P$ 的坐标是\underline{\hspace{6em}};
    \item 当$\overrightarrow{P_1P}=\lambda \overrightarrow{P_1P_2}(\lambda\neq -1)$时,点 $P$ 的坐标是\underline{\hspace{8em}}。
\end{question}
\item 已知 $\bm{a} = (4,\,2)$,与 $\bm{a}$ 垂直的单位向量的坐标为\underline{\hspace{6em}}。
\item 已知 $A(2,3)$、$B(4,-3)$,点 $P$ 在线段 $AB$ 的延长线上,且
$| \overrightarrow{AP}|=\dfrac{3}{2}\,| \overrightarrow{PB}|$,点 $P$ 的坐标为\underline{\hspace{6em}}。
\item 已知向量 $\bm{a}=(1,\,0)$、$\bm{b}=(1,\,1)$、$\bm{c}=(-1,\,0)$,$\bm{c}=\lambda\,\bm{a}+\mu\,\bm{b}$, 则 $\lambda=$ \underline{\hspace{6em}}, $\mu=$ \underline{\hspace{6em}}.
\item 若 $\bm{e}_{1}$、$\bm{e}_{2}$ 是夹角为 $60^\circ$ 的两个单位向量,则
$
\bm{a}=2\bm{e}_{1}+\bm{e}_{2}, \quad
\bm{b}=-3\bm{e}_{1}+2\bm{e}_{2}
$
的夹角为\underline{\hphantom{3cm}}。


\end{question}





% \mycenter{第二部分(非选择题\ 共\textmd{\rmfamily 110}分)}

\section{证明题:本题共$10$小题,$70$分。}
\begin{question}[resume]
    % 11
    \item (10分) 根据平面向量运算的定义,证明:对于向量 $\bm{a}, \bm{b}, \bm{c}$ 和实数 $\lambda$,有
    \[
    \begin{cases}
    (1) & \bm{a} \cdot \bm{b} = \bm{b} \cdot \bm{a}, \\[0.3em]
    (2) & (\lambda \bm{a}) \cdot \bm{b} = \lambda (\bm{a} \cdot \bm{b}) = \bm{a} \cdot (\lambda \bm{b}), \\[0.3em]
    (3) & (\bm{a} + \bm{b}) \cdot \bm{c} = \bm{a} \cdot \bm{c} + \bm{b} \cdot \bm{c}.
    \end{cases}
    \]\vspace{15em}
    
    % 12
    \item 如图,在任意四边形 $ABCD$ 中,$E,F$ 分别为 $AD,\,BC$ 的中点,求证:$\overrightarrow{AB}+\overrightarrow{DC}=2\,\overrightarrow{EF}.$

    \begin{tikzpicture}[scale=1.2]
    % 坐标点
    \coordinate (A) at (-0.1,1.8);
    \coordinate (B) at (-1,0);
    \coordinate (C) at (3,0);
    \coordinate (D) at (2,2.1);
    \coordinate (E) at ($(A)!0.5!(D)$); % AD 中点
    \coordinate (F) at ($(B)!0.5!(C)$); % BC 中点

    % 四边形边
    \draw (A) -- (B) -- (C) -- (D) -- cycle;

    % EF 向量(蓝色箭头)
    \draw[->, thick, blue] (E) -- (F);

    % 节点文字
    \node[above left]  at (A) {A};
    \node[below left]  at (B) {B};
    \node[below right] at (C) {C};
    \node[above right] at (D) {D};
    \node[above]       at (E) {E};
    \node[below]       at (F) {F};
    \node[below=6pt] at (current bounding box.south) {\small(第15题)};
    \end{tikzpicture}
    \vspace{2em}
    
    % 13
    \item 如图,$CD$ 是 $\triangle ABC$ 的中线;且 $CD=\dfrac12\,AB$。用向量方法证明 $\triangle ABC$ 是直角三角形。

    \begin{tikzpicture}[scale=1.0, line cap=round, line join=round]
    % 顶点
    \coordinate (A) at (0,0);
    \coordinate (B) at (5,0);
    \coordinate (C) at (3.3,2.6);
    \coordinate (D) at ($(A)!0.5!(B)$); % AB 中点

    % 边
    \draw (A)--(C)--(B)--cycle;              % 三角形边
    \draw[very thick, blue] (A)--(B);        % 底边 AB(蓝色加粗)
    \draw[very thick, magenta] (C)--(D);     % 中线 CD(品红加粗)

    % 顶点标注
    \node[below left]  at (A) {A};
    \node[below right] at (B) {B};
    \node[above]       at (C) {C};
    \node[below]       at (D) {D};

    % 图注(放在图的正下方中间)
    \node[below=6pt] at (current bounding box.south) {\small(图 6.3-5)};
    \end{tikzpicture}

    
    \item 用向量法证明:直径所对的圆周角是直角.\vspace{15em}
    \item 用向量方法证明两角差的余弦公式
    \[
    \cos(\alpha - \beta) = \cos\alpha \cos\beta + \sin\alpha \sin\beta.
    \]\vspace{15em}

    
    % 15
    \item 用向量方法证明:对于任意的 $a,b,c,d\in\bm{R}$,恒有不等式
    \[
    (ac+bd)^2 \leq (a^2+b^2)\,(c^2+d^2).
    \]\vspace{15em}
    \item 
    \begin{question}
        \item 用向量法证明余弦定理: $c^2=a^2+b^2-2ab\cos C$.
        \item 用向量法证明正弦定理: $$\frac{a}{\sin A}=\frac{b}{\sin B}=\frac{c}{\sin C}.$$
        \item 根据正弦定理证明:$$S_{\triangle ABC}=\frac{1}{2}ab\sin C.$$
        \item 证明:设三角形的外接圆半径为 $R$,则
        \[
        \frac{a}{\sin A}=2R.
        \]
        \item 在 $\triangle ABC$ 中,求证:\quad
        \( c\bigl(a\cos B - b\cos A\bigr)=a^{2}-b^{2}. \)

    \end{question}\vspace{30em}

    \item 在 $\triangle ABC$ 中,$BE$ 和 $CF$ 分别是两条中线,交于点 $O$;
    $D$ 为边 $BC$ 的中点。证明:$A$、$O$、$D$ 三点共线,且 $ AO = 2\,OD.$

    \begin{tikzpicture}[scale=1.2, line cap=round, line join=round]
    % 顶点
    \coordinate (A) at (1.2,3.0);
    \coordinate (B) at (0,0);
    \coordinate (C) at (4.2,0);

    % 边
    \draw[very thick] (A)--(B)--(C)--cycle;

    % 选点
    \coordinate (D) at ($(B)!0.5!(C)$);      % BC 上一点 D
    \coordinate (E) at ($(A)!0.5!(C)$);      % AC 上一点 E
    \coordinate (F) at ($(A)!0.5!(B)$);      % AB 上一点 F

    % 线段/箭线
    \draw[very thick, magenta] (B)--(E);
    \draw[very thick, magenta] (C)--(F);
    \draw[dashed, ultra thick, cyan!60] (A)--(D);

    % 交点 O
    \coordinate (O) at (intersection of B--E and C--F);
    \fill (O) circle(1pt);

    % 标注
    \node[above]      at (A) {A};
    \node[below left] at (B) {B};
    \node[below right]at (C) {C};
    \node[below]      at (D) {D};
    \node[right]      at (E) {E};
    \node[left]       at (F) {F};
    \node[right]      at (O) {O};

    % 图注(放在图的正下方中间)
    \node[below=6pt] at (current bounding box.south) {\small(第**题)};
    \end{tikzpicture}
    \vspace{10em}

\end{question}


\section{计算题:本题共\textmd{\rmfamily 80}分。}
\begin{question}[resume]
    % 1
    \item 选择题 \\
    

    % 2
    \item 选择题 \\
    
\end{question}
\end{document}
