\documentclass[12pt, a4paper]{article}
\usepackage{amsmath, enumitem, fancyhdr, geometry, fontspec, lastpage, titlesec}
\usepackage{xeCJK}
\usepackage{CJKnumb}


% 页边距
\geometry{
    left=2cm,
    right=2cm,
    top=2cm,
    bottom=2cm
}
% 字体
\setmainfont{Times New Roman}
\setCJKmainfont{"FZShuSong_GB18030-Z01"}
% section格式
\renewcommand{\thesection}{\CJKnumber{\value{section}}、}

% 自定义命令
\def\mycenter#1{\begin{center} \bf \sffamily \large #1 \end{center}}

% 页码
\pagestyle{fancy}
\fancyhf{}
\cfoot{数学试卷\ 第\thepage 页\ 共\pageref{LastPage}\ 页}
\renewcommand{\headrulewidth}{0pt}

% 题目格式. III太宽了,稍作调整
\newlist{question}{enumerate}{2}

% 北京、天津等地,(1), (2), ...
% \setlist[question,1]{
%     label=\protect{\makebox[1.6em]{(\hfill\arabic*\hfill)}},
%     labelwidth=1.6em,
%     labelsep=0.4em,
%     leftmargin=2em,
% }

% 常见格式,1., 2., ...
\setlist[question,1]{
    label=\protect{\makebox[1.6em]{\hfill\arabic*.\hfill}},
    labelwidth=1.6em,
    labelsep=0.4em,
    leftmargin=2em,
}

\setlist[question, 2]{
    topsep=0pt,
    label=\protect{\makebox[1.6em]{(\hfill\Roman*\hfill)}},
    labelwidth=1.6em,
    labelsep=0.4em,
    leftmargin=2em,
}


\begin{document}



\begin{center}
    % \huge 某某学校xxxx-xxxx学年xx考试 \\
    \huge 长安教育暑假班中期考试 \\
    \Large \bf \sffamily 数学试卷-平面向量部分
\end{center}

本试卷共\pageref{LastPage}\ 页,150分。考试时长120分钟。测试分两个部分:计算和证明。计算题仅给出最后结果不给分。
% 考生务必将答案答在答题纸上,在试卷上作答无效。考试结束后,将本试卷和答题纸一并交回。

\mycenter{第一部分(计算题\ 共\textmd{\rmfamily 80}分)}

% \section{计算题:本题共\textmd{\rmfamily 10}小题,每小题\textmd{\rmfamily 4}分,共\textmd{\rmfamily 40}分。在每小题给出的四个选项中,只有一项是符合题目要求的。}
\section{计算题:本题共\textmd{\rmfamily 80}分。}
\begin{question}[resume]
    % 1
    \item 选择题 \\
    \begin{tabular*}{\linewidth}{@{\extracolsep{\fill}}lllll@{}}
        A. 1 &
        B. 2 &
        C. 3 &
        D. 4 &
    \end{tabular*}

    % 2
    \item 选择题 \\
    \begin{tabular*}{\linewidth}{@{\extracolsep{\fill}}lllll@{}}
        A. 1 &
        B. 2 &
        C. 3 &
        D. 4 &
    \end{tabular*}

    % 3
    \item 选择题 \\
    \begin{tabular*}{\linewidth}{@{\extracolsep{\fill}}lllll@{}}
        A. 1 &
        B. 2 &
        C. 3 &
        D. 4 &
    \end{tabular*}

    % 4
    \item 选择题 \\
    \begin{tabular*}{\linewidth}{@{\extracolsep{\fill}}lllll@{}}
        A. 1 &
        B. 2 &
        C. 3 &
        D. 4 &
    \end{tabular*}

    % 5
    \item 选择题 \\
    \begin{tabular*}{\linewidth}{@{\extracolsep{\fill}}lll@{}}
        A. 1 &
        B. 2 & \\
        C. 3 &
        D. 4 &
    \end{tabular*}

    % 6
    \item 选择题 \\
    \begin{tabular*}{\linewidth}{@{\extracolsep{\fill}}lll@{}}
        A. 1 &
        B. 2 & \\
        C. 3 &
        D. 4 &
    \end{tabular*}

    % 7
    \item 选择题 \\
    \begin{tabular*}{\linewidth}{@{\extracolsep{\fill}}lll@{}}
        A. 1 &
        B. 2 & \\
        C. 3 &
        D. 4 &
    \end{tabular*}

    % 8
    \item 选择题 \\
    \begin{tabular*}{\linewidth}{@{\extracolsep{\fill}}lll@{}}
        A. 1 &
        B. 2 & \\
        C. 3 &
        D. 4 &
    \end{tabular*}

    % 9
    \item 选择题 \\
    \begin{tabular*}{\linewidth}{@{\extracolsep{\fill}}ll@{}}
        A. 1 & \\
        B. 2 & \\
        C. 3 & \\
        D. 4 &
    \end{tabular*}

    % 10
    \item 选择题 \\
    \begin{tabular*}{\linewidth}{@{\extracolsep{\fill}}ll@{}}
        A. 1 & \\
        B. 2 & \\
        C. 3 & \\
        D. 4 &
    \end{tabular*}
\end{question}

\mycenter{第二部分(非选择题\ 共\textmd{\rmfamily 110}分)}

\section{填空题:本题共\textmd{\textrm{5}}小题,每小题\textmd{\textrm{5}}分,共\textmd{\textrm{25}}分。}
\begin{question}[resume]
    % 11
    \item 填空题\underline{\hspace{3em}}.
    
    % 12
    \item 填空题\underline{\hspace{3em}}.
    
    % 13
    \item 填空题.
    \begin{question}
        \item 第一问\underline{\hspace{3em}};
        \item 第二问\underline{\hspace{3em}}.
    \end{question}
    
    % 14
    \item 填空题\underline{\hspace{3em}}.
    
    % 15
    \item 填空题\underline{\hspace{3em}}.
\end{question}

\section{解答题:本题共\textmd{\textrm{6}}小题,共\textmd{\textrm{85}}分。解答题应写出文字说明、证明过程或演算步骤。}
\begin{question}[resume]
    % 16
    \item (本小题13分) \\
    解答题
    \begin{question}
        \item 第一问;
        \item 第二问.
    \end{question}
    
    % 17
    \item (本小题13分) \\
    解答题
    \begin{question}
        \item 第一问;
        \item 第二问.
        \item 第三问.
    \end{question}
    
    % 18
    \item (本小题14分) \\
    解答题
    \begin{question}
        \item 第一问;
        \item 第二问;
        \item 第三问.
    \end{question}
    
    % 19
    \item (本小题15分) \\
    解答题
    \begin{question}
        \item 第一问;
        \item 第二问;
    \end{question}
    
    % 20
    \item (本小题15分) \\
    解答题
    \begin{question}
        \item 第一问;
        \item 第二问;
        \item 第三问.
    \end{question}
    
    % 21
    \item (本小题15分) \\
    解答题
    \begin{question}
        \item 第一问;
        \item 第二问;
        \item 第三问.
    \end{question}
\end{question}

\end{document}
