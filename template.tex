\documentclass[12pt, a4paper]{article}
\usepackage{amsmath, enumitem, fancyhdr, fontspec, geometry, lastpage, titlesec}
\usepackage[fontset = founder]{ctex}

% 页边距
\geometry{
    left=2cm,
    right=2cm,
    top=2cm,
    bottom=2cm
}

% section格式
\renewcommand{\thesection}{\chinese{section}}
\titleformat{\section}{\normalsize \bf \sffamily}{\thesection、}{0pt}{}

% 自定义命令
\def\mycenter#1{\begin{center} \bf \sffamily \large #1 \end{center}}

% 页码
\pagestyle{fancy}
\fancyhf{}
\cfoot{数学试卷\ 第\thepage 页\ 共\pageref{LastPage}\ 页}
\renewcommand{\headrulewidth}{0pt}

% 题目格式
\newlist{question}{enumerate}{2}
\setlist[question,1]{
    label=\arabic*.
}
\setlist[question, 2]{
    topsep=0pt,
    label=\protect{\makebox[2.5em]{(\hfill\Roman*\hfill)}},
    labelsep=0pt
}

\begin{document}

\setmainfont{Times New Roman}

\begin{center}
    \large 某某学校2024-2025学年第二学期期中练习 \\
    \begin{tabular}{@{}p{0.4\textwidth} c p{0.4\textwidth}@{}}
        & \Large \bf \sffamily 数学试卷 & \hfill \small 2025.04
    \end{tabular}
\end{center}

本试卷共\pageref{LastPage}\ 页,150分。考试时长120分钟。考生务必将答案答在答题纸上,在试卷上作答无效。考试结束后,将本试卷和答题纸一并交回。

\mycenter{第一部分(选择题\ 共\textmd{\rmfamily 40}分)}

\section{选择题:本题共\textmd{\rmfamily 10}小题,每小题\textmd{\rmfamily 4}分,共\textmd{\rmfamily 40}分。在每小题给出的四个选项中,只有一项是符合题目要求的。}
\begin{question}[resume]
    \item 选择题 \\
    \noindent
    \begin{tabular*}{\linewidth}{@{\extracolsep{\fill}}lllll@{}}
        A. 1 &
        B. 2 &
        C. 3 &
        D. 4 &
    \end{tabular*}

    \item 选择题 \\
    \noindent
    \begin{tabular*}{\linewidth}{@{\extracolsep{\fill}}lll@{}}
        A. 1 &
        B. 2 & \\
        C. 3 &
        D. 4 &
    \end{tabular*}

    \item 选择题 \\
    \noindent
    \begin{tabular*}{\linewidth}{@{\extracolsep{\fill}}ll@{}}
        A. 1 & \\
        B. 2 & \\
        C. 3 & \\
        D. 4 &
    \end{tabular*}
\end{question}

\mycenter{第二部分(非选择题\ 共\textmd{\rmfamily 110}分)}

\section{填空题:本题共\textmd{\textrm{5}}小题,每小题\textmd{\textrm{5}}分,共\textmd{\textrm{25}}分。}
\begin{question}[resume]
    \item 填空题\underline{\hspace{3em}}.
    \item 填空题.
    \begin{question}
        \item 第一问\underline{\hspace{3em}};
        \item 第二问\underline{\hspace{3em}}.
    \end{question}
\end{question}

\section{解答题:本题共\textmd{\textrm{6}}小题,共\textmd{\textrm{85}}分。解答题应写出文字说明、证明过程或演算步骤。}
\begin{question}[resume]
    \item (本小题15分) \\
    解答题
    \begin{question}
        \item 第一问;
        \item 第二问.
    \end{question}
    \item (本小题15分) \\
    解答题
    \begin{question}
        \item 第一问;
        \item 第二问;
        \item 第三问.
    \end{question}
\end{question}

\end{document}

